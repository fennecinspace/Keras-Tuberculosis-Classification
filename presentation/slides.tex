\documentclass{beamer}
\usepackage[utf8]{inputenc}
\usepackage{tikz}

\usetheme{metropolis}
\graphicspath{./resources/}

\title{
    \textbf{
        Tenserflow et Keras pour le Deep learning (Apprentissage profond)
    }
}

\date{\scriptsize 02-12-2018}
\author{
    \textbf{Benkedadra Mohamed}\\
    \textbf{Benkorreche Mohammed El Amine}\\
    \textbf{Youcefi Mohammed Yassine}
}

\institute{Universit\'e de Mostaganem}



\begin{document}
    \metroset{block = fill}

    \begin{frame}
        \titlepage \thispagestyle{empty}
    \end{frame}
    
    \begin{frame}{Contenu}
        \tableofcontents
    \end{frame}

    \addtocontents{toc}{Deep Learning\\}
    \begin{frame}{Deep Learning}
        \begin{itemize}
            \item Fait partie du Machine Learning
            % \item Compose de plusieurs architectures (m\'ethodes):
            % \begin{itemize}
            %     \item Deep Neural Networks
            %     \item Deep Belief Networks 
            %     \item Computer Vision
            %     \\... etc
            % \end{itemize}
            \item Inspiré par le cerveau humain
            \item Appliqué sur :
            \begin{itemize}
                \item Vision Artificielle
                \item Reconnaissance Automatique de la Parole 
                \item Traitement Automatique du Langage Naturel
                \\... etc
            \end{itemize}
        \end{itemize}
    \end{frame} 

    \begin{frame}{Deep Learning}
        \def\layersep{2.5cm}
        \begin{tikzpicture}[shorten >=1pt,->,draw=black!50, node distance=\layersep]
            \tikzstyle{every pin edge}=[<-,shorten <=1pt]
            \tikzstyle{neuron}=[circle,fill=black!25,minimum size=17pt,inner sep=0pt]
            \tikzstyle{input neuron}=[neuron, fill=green!50];
            \tikzstyle{output neuron}=[neuron, fill=red!50];
            \tikzstyle{hidden neuron}=[neuron, fill=blue!50];
            \tikzstyle{annot} = [text width=4em, text centered]

            % Draw the input layer nodes
            \foreach \name / \y in {1,...,4}
            % This is the same as writing \foreach \name / \y in {1/1,2/2,3/3,4/4}
                % \node[input neuron, pin=left:Input \#\y] (I-\name) at (0,-\y) {};
                \node[input neuron, pin=left:Attribue \#\y] (I-\name) at (0,-\y) {};

            % Draw the hidden layer nodes
            \foreach \name / \y in {1,...,5}
                \path[yshift=0.5cm]
                    node[hidden neuron] (H-\name) at (\layersep,-\y cm) {};

            % Draw the output layer node
            \node[output neuron,pin={[pin edge={->}]right:Prédiction}, right of=H-3] (O) {};

            % Connect every node in the input layer with every node in the
            % hidden layer.
            \foreach \source in {1,...,4}
                \foreach \dest in {1,...,5}
                    \path (I-\source) edge (H-\dest);

            % Connect every node in the hidden layer with the output layer
            \foreach \source in {1,...,5}
                \path (H-\source) edge (O);

            % Annotate the layers
            \node[annot,above of=H-1, node distance=1cm] (hl) {Hidden layer};
            \node[annot,left of=hl] {Input layer};
            \node[annot,right of=hl] {Output layer};
        \end{tikzpicture} 
    \end{frame} 


    \addtocontents{toc}{Tensorflow\\}
    \begin{frame}{Tensorflow}
        \begin{block}{D\'efinition}
            \vspace{0.5 em}
            Yess \only<1>{\line(1,0){20}}\only<2>{hello} Mate
        \end{block}
    \end{frame} 

    \addtocontents{toc}{Keras\\}
    \begin{frame}{Keras}
        hello
    \end{frame}

\end{document}
